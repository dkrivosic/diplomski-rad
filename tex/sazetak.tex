\begin{sazetak}

Problem detekcije objekata je jedan od najraširenijih problema u području računalnog vida, a u ovom radu se koncentriramo specifično na detekciju ljudi. Započinjemo teorijskim pregledom algoritama za detekciju gdje prvo opisujemo algoritam koji se ne temelji na konvolucijskim mrežama. Zatim detaljnije opisujemo nekoliko danas popularnih konvolucijskih modela i uspoređujemo ih na problemu detekcije ljudi. Modele uspoređujemo na nekoliko skupova podataka, a osim preciznosti, komentiramo i najčešće pogreške svakog modela uz primjere. Na kraju donosimo zaključke na temelju dobivenih rezultata.

\kljucnerijeci{Detekcija objekata, detekcija ljudi, računalni vid, duboko učenje, konvolucijske neuronske mreže}
\end{sazetak}