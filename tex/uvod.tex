Problem detekcije objekata odnosi se na pronalaženje objekta na slici, tj. određivanje lokacije objekta, te svrstavanje objekta u jedan od predefiniranih razreda. Lokacija objekta se najčešće prikazuje najmanjim pravokutnikom unutar kojeg se nalazi cijeli objekt.
U ovom radu su algoritmi za detekciju objekata podijeljeni u dvije skupine: algoritmi za detekciju koji nisu temeljeni na konvolucijskim mrežama i algoritmi za detekciju temeljeni na konvolucijskim mrežama.
Na početku su opisani algoritmi za detekciju objekata koji nisu temeljeni na konvolucijskim mrežama te je detaljnije opisan jedan algoritam kao predstavnik skupine.
Budući da je većina danas popularnih algoritama za detekciju objekata temeljena na konvolucijskim mrežama, u radu se opisuju konvolucijske mreže te operacije konvolucije i sažimanja koje su sastavni dijelovi konvolucijskih mreža. Zatim je detaljnije opisano nekoliko danas popularnih algoritama za detekciju objekata. 
U poglavlju "Opis alata", opisani su programski alati korišteni za provođenje eksperimenata i računalni resursi na kojima su se algoritmi izvodili.
Nakon teorijskog uvoda, algoritmi temeljeni na konvolucijskim mrežama su primijenjeni na zadatak detekcije osoba na označenim skupovima podataka i dan je pregled i usporedba rezultata i performansi algoritama.
Na kraju razmatramo detekciju igrača na snimkama nogometnih utakmica i komentiramo dobivene rezultate.