YOLO (You Only Look Once), \cite{DBLP:journals/corr/RedmonDGF15}, je inspiriran načinom na koji ljudi prepoznaju objekte. Ljudima je dovoljno da jednom pogledaju sliku i odmah znaju koji objekti se nalaze na slici i njihove lokacije što im omogućava obavljanje kompleksnih zadataka, kao što je vožnja, bez puno razmišljanja. 
Drugi detektori objekata koriste klasifikatore koje evaluiraju na različitim dijelovima slike. Neki koriste klizeći prozor pa klasifikator evaluiraju na svakoj lokaciji u slici, dok drugi koriste različite metode za predlaganje regija. YOLO detekciju objekata postavlja kao problem regresije i odjednom predviđa lokaciju i razred objekta iz piksela slike na ulazu. Model se sastoji od konvolucijske mreže s 24 konvolucijska sloja i na izlazu 2 potpuno povezana sloja. Izlazni sloj mreže koristi linearnu aktivacijsku funkciju, a ostali slojevi koriste funkciju:
\[
	\phi (x) = 
	\begin{cases}
		x, \quad \quad za \quad x > 0 \\
		0.1x, \quad inace
	\end{cases}
\]
 Arhitektura mreže prikazana je na slici \ref{yolo_mreza}. Mreža na ulazu očekuje sliku dimenzija $448 \times 448$ pa se slici prije detekcije treba promijeniti veličina ako nije zadane veličine. Slika na ulazu se dijeli na mrežu dimenzija $S \times S$. Ako sredina objekta pada u pojedinu ćeliju mreže, ta ćelija je odgovorna za detekciju tog objekta. Svaka ćelija predviđa $B$ prozora i za svaki predviđa vjerojatnost da se u tom prozoru nalazi objekt. Svaka predikcija prozora se sastoji od 5 brojeva: $x$ i $y$ koordinate središta, relativne visine i širine i vjerojatnosti da se u prozoru nalazi objekt. Budući da je svaka ćelija odgovorna za predviđanje jednog objekta, vjerojatnosti pripadnosti svakom od $C$ razreda se predviđaju jednom za svaku ćeliju, bez obzira na parametar $B$. Uz zadane parametre, izlaz YOLO mreže je tenzor dimenzija $S \times S \times (B \ast 5 + C)$. Višestruke detekcije se rješavaju tako da se prvo odbace sve detekcije čija je pouzdanost ispod zadanog praga. Nakon toga se za svaki razred uzima detekcija s maksimalnom pouzdanošću i odbacuju se sve detekcije s manjom pouzdanošću koje imaju velika preklapanja (omjer presjeka i unije veći od 0.5) s odabranom detekcijom. Postupak se ne ponavlja dok ne ostanu samo detekcije koje nemaju omjer presjeka i unije s detekcijama istog razreda veći od 0.5. Detekcija objekata YOLO modelom prikazana je na slici \ref{yolo_detekcija}. Za treniranje mreže se koristi sljedeća funkcija pogreške:
\begin{multline}
 	\lambda_{coord} \sum\limits_{i=0}^{S^2} \sum\limits_{j=0}^{B} \mathbbm{1}_{ij}^{obj}[(x_i - \hat{x}_i )^2 + (y_i - \hat{y_i})^2] \\
 	+ \lambda_{coord} \sum\limits_{i=0}^{S^2} \sum\limits_{j=0}^{B} \mathbbm{1}_{ij}^{obj} [(\sqrt{w_i} - \sqrt{\hat{w}_i})^2 + (\sqrt{h_i} - \sqrt{\hat{h}_i})^2] \\
 	+ \sum\limits_{i=0}^{S^2} \sum\limits_{j=0}^{B} \mathbbm{1}_{ij}^{obj}(C_i - \hat{C}_i)^2 \\
 	+ \lambda_{noobj} \sum\limits_{i=0}^{S^2} \sum\limits_{j=0}^{B} \mathbbm{1}_{ij}^{noobj}(C_i - \hat{C}_i)^2 \\
 	+ \sum\limits_{i=0}^{S^2} \mathbbm{1}_{i}^{obj} \sum\limits_{c \ \epsilon \ razredi} (p_i(c) - \hat{p}_i(c))^2
\end{multline}
gdje $\mathbbm{1}_{i}^{obj}$ označava nalazi li se objekt u ćeliji $i$, a $\mathbbm{1}_{ij}^{obj}$ je li $j$-ti prozor u ćeliji $i$ zadužen za predviđanje objekta. Funkcija pogreške kažnjava pogreške klasifikacije ako se objekt nalazi u ćeliji i kažnjava pogrešku u predviđenim koordinatama samo ako je prozor zadužen za predviđanje objekta. Budući da bi kažnjavanjem visine i širine direktno, jače kažnjavala pogreške kod detekcije velikih objekata, funkcija pogreške koristi korijen visine i širine za izračun pogreške. Kvadratna pogreška jednako kažnjava pogrešku klasifikacije i lokalizacije. Puno ćelija u slici ne sadrži objekt pa pouzdanost tih ćelija teži prema nuli, dok je gradijent ćelija u kojima se nalazi objekt prevelik što može uzrokovati ranu divergenciju. Taj problem se rješava tako da se poveća utjecaj pogreške predikcije koordinata prozora i smanji utjecaj pogreške klasifikacije što se postiže parametrima $\lambda_{coord}$ i $\lambda_{noobj}$. Autori postavljaju $\lambda_{coord} = 5$ i $\lambda_{noobj} = 0.5$. YOLO je u odnosu na prijašnje modele za detekciju objekata donio velika ubrzanja, ali i on ima neka ograničenja. Budući da je slika podijeljena u mrežu i svaka ćelija mreže predviđa mali broj prozora, YOLO ima problema s detekcijom malih objekata koji se pojavljuju u skupinama kao na primjer jato ptica. Također, ima poteškoća u generalizaciji u neuobičajenim okolinama i položajima objekata koji se jako razlikuju od onih viđenih u skupu podataka za trening.
 
 \begin{figure}
	\centering
	\includegraphics[scale=0.5]{img/yolo_mreza.png}
	\caption{Arhitektura YOLO mreže.}
	\label{yolo_mreza}
\end{figure}

 \begin{figure}
	\centering
	\includegraphics[scale=0.8]{img/yolo_detekcija.png}
	\caption{Prikaz detekcije objekata YOLO modelom.}
	\label{yolo_detekcija}
\end{figure}