\nocite{Goodfellow-et-al-2016}

Konvolucijske neuronske mreže su posebna vrsta neuronskih mreža za obradu podataka koji imaju strukturu rešetke. Podaci koji su primjereni za obradu konvolucijskim mrežama mogu biti jednodimenzionalne rešetke, kao na primjer zvuk ili očitanja senzora u određenim vremenskim intervalima, dvodimenzionalne rešetke, kao što su slike, ili trodimenzionalne rešetke, kao što su video zapisi.
Ime konvolucijskih neuronskih mreža dolazi od matematičke operacije konvolucije koja se u njima koristi pa se i konvolucijske mreže mogu definirati kao neuronske mreže koje u barem jednom sloju umjesto množenja matrica koriste konvoluciju.
U nastavku su opisane operacije konvolucije i sažimanja, koja se u praksi često koristi zajedno s konvolucijom.
%TODO: Napisati o čemu ću još pričati

\subsection{Konvolucija}
