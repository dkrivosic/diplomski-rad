Detekcija ljudi ima široku primjenu u stvarnom svijetu. U ovom radu smo opisali neke od danas najpopularnijih konvolucijskih modela za detekciju osoba. Modele smo evaluirali na nekoliko različitih skupova podataka te prikazali i komentirali dobivene rezultate. Najbolje rezultate na fotografijama ljudi postiže Faster R-CNN pa bismo mogli zaključiti da je najpouzdaniji za detekciju ljudi. Međutim, YOLO v3 postiže najbolje rezultate na umjetničkim slikama iz čega možemo zaključiti da najbolje generalizira. Zbog toga bi YOLO v3 bio najbolji izbor kada na raspolaganju imamo mali broj primjera za trening.
Mjerenjem prosječnog vremena detekcije dolazimo do zaključka da niti jedan od korištenih algoritama nije dovoljno brz za detekciju u stvarnom vremenu s nama dostupnim računalnim resursima. Najbrže detekcije smo postigli SSD modelom kojem za svaku sliku iz INRIA Person skupa podataka u prosjeku treba nešto više od 3 sekunde. Detekcija je još sporija na snimkama nogometnih utakmica budući da se radi o slikama puno veće rezolucije. Kod slika s nogometnih utakmica smo naišli i na problem točnosti detekcije jer su ljudi na tim slikama mali u odnosu na veličinu slike. Taj problem smo riješili dijeljenjem slike na više manjih dijelova i detekcijom svake od tih slika zasebno. Međutim, tim postupkom smo usporili detekciju jer je broj prolaza kroz mrežu proporcionalan broju slika na koje smo podijelili originalnu sliku. 

U budućem radu bi se navedeni algoritmi mogli testirati korištenjem boljih grafičkih procesora čime bi se ubrzalo vrijeme detekcije. Za detekciju igrača na nogometnim utakmicama bi se, uz dostupnost dovoljnog broja grafičkih procesora, moglo pokušati podijeliti sliku na više dijelova i sve dobivene slike evaluirati paralelno, čime bi vrijeme detekcije jedne slike visoke rezolucije svelo približno na vrijeme detekcije jedne slike manje rezolucije uz dodatno vrijeme potrebno za dijeljenje originalne slike na manje dijelove. Također bi se moglo pokušati trenirati detektore koji koriste manje konvolucijske mreže za izlučivanje značajki budući da se detektira samo jedan razred. Time bi se vrijeme detekcije moglo značajno smanjiti.