Fast R-CNN je uveo velika ubrzanja u treningu i detekciji u odnosu na R-CNN i ako zanemarimo vrijeme potrebno za predlaganje regija, detektira objekte gotovo u stvarnom vremenu. Međutim, Fast R-CNN je ovisan o algoritmu predlaganja regija što mu je usko grlo u kontekstu vremena detekcije. Predlaganje regija traje otprilike koliko i detekcija objekata.
Faster R-CNN (\cite{NIPS2015_5638}) umjesto algoritama poput selektivne pretrage, koristi neuronsku mrežu za predlaganje regija koja je uz korištenje grafičkog procesora za red veličine brža od najbržih algoritama za predlaganje regija.
Mreža za predlaganje regija na ulazu prima sliku proizvoljne veličine i na izlazu vraća prijedloge regija. Mreža je potpuno konvolucijska i dijeli konvolucijske slojeve s Fast R-CNN mrežom koju Faster R-CNN koristi za detekciju objekata. Na dijeljene slojeve se dodaje jedan konvolucijski dimenzija $n \times n$ i na njega dva konvolucijska sloja dimenzija $1 \times 1$. Na izlazu konvolucijskog sloja dimenzija $n \times n$ je ReLu aktivacijska funkcija $ReLu(x) = max(0, x)$. Izlaz mreže se zatim prosljeđuje u dva izlazna sloja: sloj za klasifikaciju regije i sloj za regresiju prozora u kojem se nalazi objekt. 
Za svaku lokaciju mape značajki zadnjeg dijeljenog konvolucijskog sloja se predviđa $k$ prozora pa izlazni sloj za predviđanje prozora ima $4k$ izlaza, a klasifikacijski sloj $2k$ koji označavaju vjerojatnosti da se objekt nalazi ili ne nalazi u prozoru. Predviđene lokacije prozora se gledaju relativno u odnosu na $k$ referentnih prozora koji se još nazivaju sidra. Sidra su pravokutnici različitih, unaprijed određenih veličina i omjera stranica koji se ponavljaju na svakoj lokaciji mape značajki. Za mapu značajki dimenzija $W \times H$, ukupan broj referentnih prozora je $W \times H \times k$. Time se postiže invarijantnost modela na translaciju.
Za treniranje mreže za predlaganje regija svakom referentnom prozoru se dodjeljuje pozitivna ili negativna oznaka. Pozitivna oznaka se dodjeljuje onom prozoru koji ima najveći omjer presjeka i unije sa stvarnim prozorom u kojem se nalazi objekt i svim prozorima čiji je omjer presjeka i unije veći od 0.7. Negativne oznake se daju svim prozorima koji nisu označeni kao pozitivni, a nemaju omjer presjeka i unije s nijednim stvarnim prozorom veći od 0.3.
S tako definiranim pozitivnim i negativnim primjerima, funkcija pogreške je 
\[
	L(\{p_i\}, \{t_i\}) = \frac{1}{N_{cls}} \sum\limits_i L_{cls}(p_i, p_i^*) + \lambda \frac{1}{N_{reg}} \sum\limits_i p_i^* L_reg(t_i, t_i^*)
\]
gdje $i$ označava indeks sidra, a $p_i$ predviđenu vjerojatnost da se u tom prozoru $i$ nalazi objekt. $p_i^*$ je jednak 1 ako se objekt nalazi u prozoru, a 0 ako se ne nalazi. $t_i$ je četverodimenzionalni vektor koji označava predviđeni prozor, a $t_i^*$ označava stvarni prozor u kojem se nalazi objekt. Pogreška klasifikacija $L_{cls}$ je negativna log-izglednost razreda, a pogreška regresije $L_{reg}$ se računa kao $R(t_i - t_i^*)$, gdje je $R$ zaglađena $L_1$ pogreška koja je ista kao i kod Fast R-CNN-a. Izraz $p_i^*L_{reg}$ znači da se pogreška regresije računa samo za pozitivne primjere. Vektori $t$ i $t^*$ ne prikazuju stvarne vrijednosti koordinata prozora, visine i širine nego koriste izvedene vrijednosti:
\[
	t_x = (x - x_a) / w_a \quad t_y = (y - y_a) / h_a \quad
	t_w = log(w / w_a) \quad t_h = log(h / h_a)
\]
\[
	t_x^* = (x^* - x_a) / w_a \quad t_y^* = (y^* - y_a) / h_a \quad
	t_w^* = log(w^* / w_a) \quad t_h^* = log(h^* / h_a)
\]

Koristeći opisanu funkciju pogreške, mreža za predlaganje regija se može trenirati stohastičkim gradijentnim spustom korištenjem Backpropagation algoritma za računanje gradijenata.
Budući da mreža za predlaganje regija, i Fast R-CNN dijele konvolucijske slojeve, potrebno je trenirati mrežu tako da pogreška predlaganja regije, kao i pogreške klasifikacije i regresije utječu na težine konvolucijskih slojeva. Kako bi to bilo moguće, razvijen je algoritam koji se sastoji od četiri koraka. U prvom koraku se trenira mreža za predlaganje regija inicijalizirana modelom treniranim na skupu podataka ImageNet. U drugom koraku se trenira Fast R-CNN korištenjem predloženih regija generiranih mrežom treniranom u prvom koraku. U trećem koraku koristimo konvolucijske slojeve Fast R-CNN-a za inicijalizaciju mreže za predlaganje regija, ali se dijeljeni slojevi ne mijenjaju nego se treniraju slojevi specifični za mrežu za predlaganje regija. U četvrtom koraku dijeljeni konvolucijski slojevi su također nepromijenjeni, a treniraju se potpuno povezani slojevi Fast R-CNN mreže.
Uvođenjem mreže za predlaganje regija koja dijeli konvolucijske slojeve s Fast R-CNN-om postiže se ubrzanje kod treninga i detekcije i povećanje preciznosti detekcije.
	